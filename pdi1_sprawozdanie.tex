\documentclass{report}

\usepackage{graphicx}
\usepackage{subcaption}
\usepackage{setspace}
\usepackage[utf8]{inputenc}
\usepackage{polski}


%name and stuff
\title{Autonomiczna Nawigacja Manipulatora Dwuręcznego z Wielokierunkową Bazą Mobilną}
\date{\today}
\author{Marcin Skrzypkowski}



\begin{document}
	
	
	%customizacja strony tytułowej
	\makeatletter
	\renewcommand{\maketitle}{\begin{titlepage}
		\begin{center}{\LARGE {\bf Wydział Elektroniki i Technik Informacyjnych}}\\
			\vspace{0.4cm}
			{\LARGE {\bf Politechnika Warszawska}}\\
			\vspace{0.3cm}
		\end{center}
		\vspace{5cm}
		\begin{center}
			{\bf \LARGE Pracownia Dyplomowa Inżynierska  1\\ sprawozdanie  \vskip 0.1cm}
		\end{center}
		\vspace{1cm}
		\begin{center}
			{\bf \LARGE \@title}
		\end{center}
		\vspace{2cm}
		\begin{center}
			{\bf \Large \@author \par}
		\end{center}
		\vspace*{\stretch{1}}
		\begin{center}
			{\bf \large opiekun pracy: \\ dr Wojciech Szynkniewicz\par}
		\end{center}
		\vspace*{\stretch{5}}
		\begin{center}
			\bf{\large{Warszawa, \@date\vskip 0.1cm}}
		\end{center}
		\end{titlepage}
}
\makeatother	


globalna nawigacja:
	-
	
lokalny planner:
	-poprawa holonomiczności
	

	\pagenumbering{gobble}
	\maketitle
	\newpage



	\pagenumbering{arabic}
	\tableofcontents

	\newpage
	
	\chapter{Cel pracy}

Celem pracy jest stworzenie systemu nawigacji pozwalającego na samodzielne tworzenie mapy przez robota Velma na bazie wielokierunkowej.
Jest to jedna z części potrzebnych do otrzymania robota spełniającego warunki robota manipulacyjnego i mobilnego, które zostały przedyskuitowane w punkcie \textbf{Motywacja}. (\textbf{dodaj linka} 
Część manipulacyjna jest w zaawansowanej fazie i jest wciąż ulepszana. (\textbf{zalinkuj pracę Grześka})
Robot powinien być w stanie odnaleźć się w przestrzeni z pomocą zamontowanych czujników laserowych oraz jednostki inercyjnej. 
Robot powinien również wykrywać przeszkody niewidoczne z poziomu podłogi (stół, wystające elementy ze ścian).

Docelowo planowana jest integracja wszystkich elementów z udostępnioną wielokierunkową bazą robota. Dzięki temu manipulator na bazie będzie w stanie wykonać podane mu konkretne zadanie.

	\begin{itemize}
		\item poprawić planner globalny
		\item poprawić planner lokalny (?)
		
	\end{itemize}•

	%\chapter{Cel pracy}
		\section{Motywacja}
		Robotyka pod względem przydzielania zadań może zostać podzielona na dwie kategorie.
		Pierwszą z nich jest robotyka mobilna, gdzie zadania wymagają możliwości przemieszczania się, tworzenia mapy otoczenia, przenoszenia przedmiotów z miejsca na miejsce. 
		Drugą jest robotyka manipulacyjna, gdzie zadanie wymaga precyzyjnej często powtarzanej operacji jak przygotowanie elementów do lutowania na płytce drukowanej. 
		Połączenie bazy wielokierunkowej oraz manipulatora Velma ma połączyć możliwości obydwu kategorii zadań, zarówno tych wymagających przemieszczania ciała robota na duże odległości, jak i manipulacji w środowiskach trudnodostępnych lub wymagających precyzji.

		Zapewnienie tego połączenia zwiększy autonomię robotów, co jest celem rozwoju robotyki. (\textbf{znajdź jakieś źródło które to potwierdza})
		Dzięki temu zaistnieje możliwość zlecenia kilku zadań w różnych fizycznych przestrzeniach, które będą mogły zostać wykonane bez ingerencji człowieka. 
		Istnieje też możliwość reakcji robota na bodziec w innym pomieszczeniu, dzięki elementom mobilnym będzie  w stanie nawigować i przemieszczać się w przestrzeni, a dzięki zdolnościom manipulacyjnym wykona skomplikowane zadanie z tej dziedziny robotyki.
		
		
	\section{Opis Zadania}
\label{opis_zadania}

W sekcji \ref{cel_pracy} przedstawione zostały trzy główne problemy, które moja praca inżynierska ma na celu rozwiązać.

\subsection{Budowanie Mapy}

	W celu poprawnej lokalizacji konieczna jest uprzednio zbudowana mapa, w tym przypadku dwuwymiarowa mapa zajętości. 
	Jak już wcześniej zostało wspomniane, obecnie budowanie mapy wykorzystuje jeden czujnik laserowy, dowolny z obydwu dostępnych. 
	Docelowy system będzie w stanie wykorzystać je jednocześnie przyspieszając tworzenie mapy zajętości.
	Wykorzystanie jedynie czujników w bazie w tak wysokim robocie powoduje duże ryzyko pominięcia przeszkód znajdujących się poza ich zasięgiem, jak przykładowo stół na cienkich nogach.
	Taka przeszkoda pojawi się na mapie zajętości w postaci czterech drobnych plam, a więc istnieje ryzyko, że robot uderzy w blat stołu nie wiedząc, że ten się tam znajduje.



	W celu rozwiązania tego problemu planowana jest integracja kamery Kinect do opisanego przed chwilą systemu. 
	Trójwymiarowa reprezentacja przestrzeni z kamery zostanie zrzutowana na mapę dwuwymiarową przygotowaną przez skanery.
	Do przebadania zostanie problem blokowania niektórych pozycji docelowych bazy, przykładowo tak przygotowana mapa zajętości uniemożliwi bazie wjechanie częściowo pod blat, a taka operacja zwiększyła by zasięg manipulatorów nad powierzchnią stołu.

\subsection{Lokalizacja}


	Obecnie lokalizacja robota odbywa się jedynie z pomocą odometrii, liczonej na podstawie prędkości bazy jezdnej, lub poprzez pobieranie wiadomości bezpośrednio z symulatora Gazebo.
	Ponieważ dokładność pierwszej metody jest niska \cite{jsikora-bsc-20-twiki}, a druga nie ma prawa bytu jeżeli celem jest testowanie lokalizacji, należy wykorzystać dostępne czujniki.
	Przetestowane zostaną pojedynczy czujnik laserowy, kamera Kinect oraz jednostka inercyjna w wielu konfiguracjach, aby sprawdzić, czy najlepsze wyniki daje jeden konkretny rodzaj czujnika, czy dwa lub trzy rodzaje połączone ze sobą.

\subsection{Holonomiczność}

	Obecna implementacja
	
		
\end{document}
