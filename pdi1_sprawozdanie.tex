\documentclass{report}

\usepackage{graphicx}
\usepackage{subcaption}
\usepackage{setspace}
\usepackage[utf8]{inputenc}
\usepackage{polski}


%name and stuff
\title{Autonomiczna Nawigacja Manipulatora Dwuręcznego z Wielokierunkową Bazą Mobilną}
\date{\today}
\author{Marcin Skrzypkowski}



\begin{document}
	
	
	%customizacja strony tytułowej
	\makeatletter
	\renewcommand{\maketitle}{\begin{titlepage}
		\begin{center}{\LARGE {\bf Wydział Elektroniki i Technik Informacyjnych}}\\
			\vspace{0.4cm}
			{\LARGE {\bf Politechnika Warszawska}}\\
			\vspace{0.3cm}
		\end{center}
		\vspace{5cm}
		\begin{center}
			{\bf \LARGE Pracownia Dyplomowa Inżynierska  1\\ sprawozdanie  \vskip 0.1cm}
		\end{center}
		\vspace{1cm}
		\begin{center}
			{\bf \LARGE \@title}
		\end{center}
		\vspace{2cm}
		\begin{center}
			{\bf \Large \@author \par}
		\end{center}
		\vspace*{\stretch{1}}
		\begin{center}
			{\bf \large opiekun pracy: \\ dr Wojciech Szynkniewicz\par}
		\end{center}
		\vspace*{\stretch{5}}
		\begin{center}
			\bf{\large{Warszawa, \@date\vskip 0.1cm}}
		\end{center}
		\end{titlepage}
}
\makeatother	


globalna nawigacja:
	-
	
lokalny planner:
	-poprawa holonomiczności
	

	\pagenumbering{gobble}
	\maketitle
	\newpage



	\pagenumbering{arabic}
	\tableofcontents

	\newpage
	
	\chapter{Cel pracy}
	blabla

	%\chapter{Cel pracy}
		\chapter{Motywacja}
		bla
		
	\section{Opis Zadania}
\label{opis_zadania}

W sekcji \ref{cel_pracy} przedstawione zostały trzy główne problemy, które moja praca inżynierska ma na celu rozwiązać.

\subsection{Budowanie Mapy}

	W celu poprawnej lokalizacji konieczna jest uprzednio zbudowana mapa, w tym przypadku dwuwymiarowa mapa zajętości. 
	Jak już wcześniej zostało wspomniane, obecnie budowanie mapy wykorzystuje jeden czujnik laserowy, dowolny z obydwu dostępnych. 
	Docelowy system będzie w stanie wykorzystać je jednocześnie przyspieszając tworzenie mapy zajętości.
	Wykorzystanie jedynie czujników w bazie w tak wysokim robocie powoduje duże ryzyko pominięcia przeszkód znajdujących się poza ich zasięgiem, jak przykładowo stół na cienkich nogach.
	Taka przeszkoda pojawi się na mapie zajętości w postaci czterech drobnych plam, a więc istnieje ryzyko, że robot uderzy w blat stołu nie wiedząc, że ten się tam znajduje.



	W celu rozwiązania tego problemu planowana jest integracja kamery Kinect do opisanego przed chwilą systemu. 
	Trójwymiarowa reprezentacja przestrzeni z kamery zostanie zrzutowana na mapę dwuwymiarową przygotowaną przez skanery.
	Do przebadania zostanie problem blokowania niektórych pozycji docelowych bazy, przykładowo tak przygotowana mapa zajętości uniemożliwi bazie wjechanie częściowo pod blat, a taka operacja zwiększyła by zasięg manipulatorów nad powierzchnią stołu.

\subsection{Lokalizacja}


	Obecnie lokalizacja robota odbywa się jedynie z pomocą odometrii, liczonej na podstawie prędkości bazy jezdnej, lub poprzez pobieranie wiadomości bezpośrednio z symulatora Gazebo.
	Ponieważ dokładność pierwszej metody jest niska \cite{jsikora-bsc-20-twiki}, a druga nie ma prawa bytu jeżeli celem jest testowanie lokalizacji, należy wykorzystać dostępne czujniki.
	Przetestowane zostaną pojedynczy czujnik laserowy, kamera Kinect oraz jednostka inercyjna w wielu konfiguracjach, aby sprawdzić, czy najlepsze wyniki daje jeden konkretny rodzaj czujnika, czy dwa lub trzy rodzaje połączone ze sobą.

\subsection{Holonomiczność}

	Obecna implementacja planera lokalnego nie wspiera ruchu bazy we wszystkich kierunkach, po zadaniu pozycji planer globalny poprawnie wyznacza ścieżkę, lecz planer lokalny steruje robotem jak bazą różnicową, nie wykorzystując większej ilości stopni swobody bazy. 
	Docelowo należy dostroić dostarczony już planer lokalny aby wykorzystywał pełnię możliwości platformy, lub, jeśli ten sposób zawiedzie, znaleźć i zaimplementować inny, który zapewni wymaganą specyfikację.
	
		
\end{document}
