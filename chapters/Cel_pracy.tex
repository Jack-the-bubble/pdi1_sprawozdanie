\section{Cel pracy}

Możliwość poruszania się robota jest wymaganiem wielu współczesnych aplikacji \cite{auto_slam}.  
Niniejsza praca skupia się na implementacji algorytmów autonomicznej lokalizacji, planowania ruchu robota Velma na bazie wielokierunkowej oraz wykonywania zadanych trajektorii. 
Ponieważ problem planowania części manipulacyjnej jest rozpatrywany w pracy Grzegorza Fijałkowskiego, nie jest on tu poruszany.
Elementy przedstawione w następnych rozdziałach dotyczą ruchu w przestrzeni zamkniętej, można je podzielić na dwa główne aspekty.
Pierwszym z nich jest lokalizacja.
%TODO: ogarnij co to za lidary
%TODO: dodaj link do kinecta
W mojej pracy inżynierskiej zostaną przedstawione prace mające na celu wykorzystanie obydwu czujników laserowych do jednoczesnej budowy dwuwymiarowej mapy zajętości oraz wykorzystanie kamery Kinect w dodawaniu do niej elementów znajdujących się powyżej zasięgu pozostałych czujników. 

Dodatkowym problemem lokalizacji, który zostanie poruszony, jest badanie jej dokładności w zależności od wykorzystanych czujników.
Badania mają odpowiedzieć na pytanie, które czujniki najlepiej w tym celu wykorzystywać.

Drugim aspektem jest poruszanie robotem holonomicznym. W obecnej chwili baza jest w stanie poruszać się we wszystkich kierunkach, a moim zadaniem jest przeprowadzenie automatyzacji działań potrzebnych do osiągnięcia tego efektu przez dodanie odpowiedniego algorytmu planowania.

Punkty przedstawione powyżej są rozwinięciem istniejącej platformy, przygotowanej przez studentów Politechniki Warszawskiej.
%TODO: prace ludzi robiących Velmę, Velmobil i połączenie.
Celem dalszego rozwoju jest stworzenie platformy spełniającej współczesne standardy autonomicznych robotów zarówno mobilnych, jak i manipulacyjnych, ale, jak zostało wspomniane powyżej, moja praca jest skupiona wokół mobilnych aspektów platformy.
