\chapter{Cel pracy}

Celem pracy jest stworzenie systemu nawigacji pozwalającego na samodzielne tworzenie mapy przez robota Velma na bazie wielokierunkowej.
Jest to jedna z części potrzebnych do otrzymania robota spełniającego warunki robota manipulacyjnego i mobilnego, które zostały przedyskuitowane w punkcie \textbf{Motywacja}. (\textbf{dodaj linka} 
Część manipulacyjna jest w zaawansowanej fazie i jest wciąż ulepszana. (\textbf{zalinkuj pracę Grześka})
Robot powinien być w stanie odnaleźć się w przestrzeni z pomocą zamontowanych czujników laserowych oraz jednostki inercyjnej. 
Robot powinien również wykrywać przeszkody niewidoczne z poziomu podłogi (stół, wystające elementy ze ścian).

Docelowo planowana jest integracja wszystkich elementów z udostępnioną wielokierunkową bazą robota. Dzięki temu manipulator na bazie będzie w stanie wykonać podane mu konkretne zadanie.

	\begin{itemize}
		\item poprawić planner globalny
		\item poprawić planner lokalny (?)
		
	\end{itemize}•