	\section{Motywacja}
		Robotyka pod względem przydzielania zadań może zostać podzielona na dwie kategorie.
		Pierwszą z nich jest robotyka mobilna, gdzie zadania wymagają możliwości przemieszczania się, tworzenia mapy otoczenia, przenoszenia przedmiotów z miejsca na miejsce. 
		Drugą jest robotyka manipulacyjna, gdzie zadanie wymaga precyzyjnej często powtarzanej operacji jak przygotowanie elementów do lutowania na płytce drukowanej. 
		Połączenie bazy wielokierunkowej oraz manipulatora Velma ma połączyć możliwości obydwu kategorii zadań, zarówno tych wymagających przemieszczania ciała robota na duże odległości, jak i manipulacji w środowiskach trudnodostępnych lub wymagających precyzji.

		Zapewnienie tego połączenia zwiększy autonomię robotów, co jest celem rozwoju robotyki. (\textbf{znajdź jakieś źródło które to potwierdza})
		Dzięki temu zaistnieje możliwość zlecenia kilku zadań w różnych fizycznych przestrzeniach, które będą mogły zostać wykonane bez ingerencji człowieka. 
		Istnieje też możliwość reakcji robota na bodziec w innym pomieszczeniu, dzięki elementom mobilnym będzie  w stanie nawigować i przemieszczać się w przestrzeni, a dzięki zdolnościom manipulacyjnym wykona skomplikowane zadanie z tej dziedziny robotyki.
		